\documentclass{jsarticle}
\usepackage[dvipdfmx]{graphicx}
\title{レポートタイトル}

\author{学生番号XXX-XXXX アカリク太郎}
\date{\today}
\begin{document}
\maketitle
\section{Cloud LaTeXへようこそ}

Cloud LaTeXは,\LaTeX を使った文書の作成・管理をクラウド上で行えるWebサービスです.
\LaTeX を使うと,複雑な数式
\begin{equation}
  \frac{\pi}{2} =
  \left( \int_{0}^{\infty} \frac{\sin x}{\sqrt{x}} dx \right)^2 =
  \sum_{k=0}^{\infty} \frac{(2k)!}{2^{2k}(k!)^2} \frac{1}{2k+1} =
  \prod_{k=1}^{\infty} \frac{4k^2}{4k^2 - 1}
\end{equation}
を含んだ読みやすくきれいな文書作成ができます.

本サービスは,\LaTeX 文書をリアルタイムに保存・コンパイルし,ユーザーアカウント別に管理します.
そのため,本サービスにログインするだけで,どこからでも作業を再開でき,ファイルを持ち歩く必要はありません.
また,様々な \LaTeX テンプレートが用意されているので,手軽に文書を作り始めることができます.

\begin{figure}
 \centering
   \includegraphics[width=60mm]{figures/Sample.eps}
 \caption{ここにキャプションを挿入します}
 \label{fig:model}
\end{figure}

Cloud LaTeXでは,作成されるPDFそのままのレイアウトで表示するPDFビューモードがあり,コンパイル画面を確認しながら文書を作成することができます(図\ref{fig:model})
日本語では,pLaTeX / LuaLaTeX / upLaTeX でのコンパイルが可能です.
また,日本語や英語文書作成だけでなく,中国語・ハングルに対応した XeLaTeX のコンパイルも可能です.
ぜひ使ってみてください.
\end{document}
